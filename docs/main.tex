\documentclass[a4paper, 11pt,draft]{article}

\usepackage{amsmath}
\usepackage{amssymb}
\usepackage{amsthm}
\usepackage{url}
\usepackage{graphicx}
\usepackage{hyperref}


\begin{document}

\title{Contagion of an exogenous failure in a bank network}
\date{\today}
\author{Marco Tinacci}
\maketitle

All the material of this work, included the up-to-date version of this document, can be found in the public repository \url{https://github.com/marcotinacci/interbank-lending-systemic-risk}.

\section{Introduction} % (fold)
\label{sec:introduction}

% section introduction (end)

\section{Model} % (fold)
\label{sec:model}

% TODO dati reali bancari, andamenti descritti nel paper
% TODO modello casuale generato, chung lu
% TODO correlazione tra asset e degree, grafici scatter

% section model (end)

\section{Implementation} % (fold)
\label{sec:implementation}

% TODO un po' di descrizione del codice a grandi linee
% TODO uso delle funzioni networkx principali

% section implementation (end)

\section{Experiments} % (fold)
\label{sec:experiments}

% TODO grafici dati
% TODO tabelle dati
% TODO all'aumentare del numero di classi il FF diminuisce a 

% section experiments (end)

\section{Conclusions} % (fold)
\label{sec:conclusions}

% section conclusions (end)

\bibliographystyle{plain}
\bibliography{biblio}

\end{document}
